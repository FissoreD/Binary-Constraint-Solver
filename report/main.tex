\documentclass{rapport}

\usepackage{variable}
\usepackage{makePlots}


\begin{document}

\maketitle

\section{Introduction}

The goal of this project is to realize a binary constraint solver based on the arc consistency filtering algorithms \ac{3}, \ac{4}, \ac{6}, \ac{2001}. As we have seen in the Constraint Programming course provided by Mr. Régin, a constraint satisfaction problem is a problem made of a set of variables $V = \{V_1, ..., V_n\}$ such that each variable $V_i$ is defined over a domain $D_i$. A domain is a set of values that can be assigned to the corresponding variable.

A constraint is a relations between a set of values taken from some of the given domains such that each domain of each constraint appear at most one time.

Let $n$ be the number of variables of the problem, the constraint problem can be represented as an undirected $n$-partite hyper-graph $G = (V, E)$ where each partition of the graph is made of the values of each domain. $V$ is made of all the values of each variable and $E$ represent the constraints of the graph.

\begin{example}
  \label{ex:p1}
  If we have the variables $V_1 = \{1, 2, 3\}$ and $V_2 = \{1, 2\}$ and $V_3 = \{0, 1, 2\}$ and the constraint $C_1 \triangleq |v_1 - v_2| = v_3$, we build the hyper-graph $G$ made of the vertices $\{1_{V_1}, 2_{V_1}, 3_{V_1}, 1_{V_2}, 2_{V_2}, 0_{V_3}, 1_{V_3}, 2_{V_3}\}$ and the hyper-edges are made such that the constraint $C_1$ is respected, for instance we can build the multi-edge $e = \{1_{V_1}, 1_{V_2}, 0_{V_3}\}$ since the absolute value of the difference between the value $1$ from $V_1$ and the value $1$ from $V_2$ gives $0$ in $V_3$.
\end{example}

A solution of a constraint satisfaction problem is a simple-path passing exactly one time through each partition of the hyper-graph, this is an equivalent for the classic definition which says that a solution is a the choice of a value for each variable such that every constraint are satisfied.

A value $v_i \in D_i$ of the variable $V_i$ is supported in the hyper-graph if for each constraint $c_1$ involving the domains $D(C) = \{d1, \dots, d_n\}$ we have $d_i \in D(C)$ and for each domain $d_j \in D(C)$ there exists a value $v_j \in D_j$ having a relation with $v_i$. A not-supported value can be removed from its domains since it cannot be part of a solution of the problem.

We can find a solution of a constraint problem by choosing an arbitrary value $v_i$ from a domain $D_i$ and removing all the other values in $D_i$. We look for all the domains $D_j$ linked to $D_i$ and remove all the values in $D_j$ that are no more supported; these values are called \textit{delta domain}. We repeat this operation until no modification can be performed. This operation is called \textit{propagation}. If after the propagation there exists an empty domain it means that there does not exist a solution containing $v_1$. We repeat the procedure with backtracking the state of the problem before $v_i$ was chose and we choose a new value in $D_i$ different from $v_i$. If on the other hand, after propagation we have no empty domains, we take a value $v_j$ from another domain $D_j$ and repeat the procedure. If we are able to select a value for each value which not produce an empty domain, it means we have found a solution.

\section{Binary constraints and arc consistency}

An interesting property of constraint satisfaction problems is that they can always been rewritten in an equivalent problem having only binary constraint. A binary constraint is a constraint linking only two variables and thanks to this strategy the corresponding graph will have no more hyper-edges.

\begin{example}
  If we retake the problem depicted in \cref{ex:p1}, we can change its model by adding an auxiliary variable $V_{aux}$ representing the ``index'' of each multi-relation of the original problem. For example, if we take $\{1_{V_1}, 1_{V_2}, 0_{V_3}\}$, we can say that $1_{V_{aux}}$ is the index of this tuple of values. The constraint $C_1$ is split in $3$ sub-constraints: $C_1^1$ representing the link between $V_{aux}$ and $V_1$, $C_1^2$ representing the link between $V_{aux}$ and $V_2$, $C_1^3$ representing the link between $V_{aux}$ and $V_3$. Note that constraint $C_1^3$ is made in order to respect the original constraint $C_1$. A more detailed example will be provided in \cref{sec:allIntGen}
\end{example}

In the state of the art we can find a lot of algorithms aiming to filter the values of a domains returning the set of not-supported values in a binary constraint satisfaction problem after deletion of a value $v_i$ in a domain $D_i$.

In the following paragraph we will sketch the main ideas behind the algorithms \ac{3}, \ac{4}, \ac{6}, \ac{2001}.

\paragraph{\ac{3}}

In the \ac{3} algorithm, after the deletion of a value $v_i$ from a domain $D_i$, \ac{3} will iterate over each domain $D_j$ with a relation with $D_i$ and for each value of $v_j \in D_j$, if there does not exist a value in $D_i$ supporting $v_j$, $v_j$ will be returned.

\paragraph{\ac{4}}

This algorithm has an internal data structure in order to improve the search of not-supported variables. Each value of each domain is associated to the list of the values supporting it. When we remove a value $v_i \in D_i$, we can directly know which variables $v_j$ depend on $v_i$ and if $v_j$ has no other variable in $D_i$ supporting it, $v_j$ is returned.

\paragraph{\ac{6}}

In \ac{6}, the internal data structure is similar to the one of \ac{6}, but instead of associating each value $v_i$ to all the value $v_j$ supporting $v_i$, we only store the first value in each domain supporting $v_i$. In this way, we reduce the amount of data to store in memory and when a value $v_i \in D_i$ is removed, we look for the values $v_j$ supporting it and if the support of $v_j$ in $D_i$ is different from $v_i$ nothing is done. Otherwise, we look for a new support in $D_i$ starting from the value $v_i$. Note that in this algorithm it is important to give an order to the values in the domain.

\paragraph{\ac{2001}}

In order to use the minimum amount of space, \ac{2001} stores, for each value $v_i \in D_i$, the first element $v_j$ in the related domain supporting it. When $v_i$ is removed, we only look for the values depending on $v_j$ and for them we look for a new support starting from $v_i$ in $D_i$, if such value doesn't exist it means that $v_j$ can be removed from $D_j$.

\section{My Implementation}

I have developed my solver in OCaml (v. 4.13.1) a functional programming language using pointer and the \textit{Base} library since I have noticed better speed performances compared of the standard OCaml modules. In the following subsections I will provide a brief explanation of some of the most important modules I have implemented.

\subsection{Doubly linked lists}

A doubly linked list (\textit{dll}) is a list whose elements have a pointer to their corresponding following and preceding element. One can note that the predecessor (resp. successor) of the first (resp. the last) element of a doubly linked list are represented by a fictive object: in my case I have used the \textit{None} option type. Doubly linked lists are particularly useful since the insertion and the deletion of an element of a \textit{dll} can be done in constant time: this is particularly useful to backtrack a list to a previous state.

\begin{minted}{ocaml}
  type 'e node = {
    value : 'e;
    id : int;
    dll_father : 'e t;
    mutable prev : 'e node option;
    mutable next : 'e node option;
    mutable is_in : bool;
  }
  
  and 'e sentinel = { mutable first : 'e node; mutable last : 'e node }
  and 'e t = { id_dom : int; name : string; mutable content : 'e sentinel option }
\end{minted}

We can see that the type \textit{node} has a \textit{prev} and a \textit{next} field which are of type optional. The doubly linked list, itself, is represented by the type \textit{t} (following the OCaml convention) and it contains a sentinel pointing on the first and the last element of the \textit{dll}.

To represent in a unique way nodes and \textit{dll}, I have added the field \textit{id} (resp. \textit{id\_dom}) in order to find them quickly when looking inside \textit{Hash-Tables}.

The id of those records are generated through the generator:

\begin{minted}{ocaml}
  let gen =
    let x = ref 0 in
    fun () -> incr x; !x
\end{minted}

Inside the module \textit{DoublyLinkedList}, I have added all of those utility functions allowing to modify the content of a \textit{dll}. In particular, I can create, remove, insert, append or prepend a node inside a \textit{dll} modifying correspondingly the \textit{prev} and the \textit{next} fields.

Moreover, I took inspiration from the \textit{List} module in OCaml and I have added some higher-order functions in order to check if an element belongs to a \textit{dll}, if a property is verified for every element in the \textit{dll} (a kind of \textit{foreach} in Java streams), \etc.

\subsection{Graph, domains and constraints representation}

The graph, the domains and the constraints are all implemented in the \textit{Graph} module since they contains the information about the problem.

\begin{minted}{ocaml}
  open Base
  module DLL = DoublyLinkedList
  
  type 'a relation = 'a DLL.node -> 'a DLL.node -> bool
  type 'a table_type = (int * int) Hash_set.t
  type 'a domain = 'a DLL.t
  
  type 'a graph = {
    tbl : 'a table_type;
    relation : 'a relation;
    constraint_binding : (string, 'a domain DLL.t) Hashtbl.t;
    domains : (string, 'a DLL.t) Hashtbl.t;
  }
\end{minted}

The type graph is a record containing a \textit{Hash\_set} of pairs of integers, that are the \textit{id} of two values supporting each other; a \textit{relation} taking two node and returning if they are linked in the constraint graph (it can be seen as the edges of the graph). Finally the \textit{constraint\_binding} is a \textit{Hash-Table} associating to the \textit{id} of each domain the set of domain linked through a constraint.

We can add constraints between values through the auxiliary function:

\begin{minted}{ocaml}
  let add_constraint (graph : 'a graph) d1 v1 d2 v2 =
    let add_if_absent (d1 : 'a domain) (d2 : 'a domain) =
      let dom =
        Hashtbl.find_or_add graph.constraint_binding d1.id_dom
          ~default:(fun _ ->
            Hashtbl.add_exn graph.domains ~key:d1.name ~data:d1;
            DLL.empty "")
      in
      DLL.add_if_absent (fun e -> phys_equal e.value d2) d2 dom
    in
    let get d v = DLL.find_by_value v d in
    let a, b = (get d1 v1, get d2 v2) in
    Hash_set.add graph.tbl (a.id, b.id);
    add_if_absent d1 d2;
    add_if_absent d2 d1
\end{minted}

This function take in parameter a graph, the name of a variable $v_1$ followed by the name of its domain $d_1$ and a second variable $v_2$ with the name of its corresponding domain $d_2$. In this function, $d_1$ and $d_2$ are inserted to the list of domains and the constraint between the node $v_1$ and $v_2$ is added.

\subsection{The solver}

The solver is the engine behind the resolution of a CP problem. The solver select the values of the domains and each time a selection is performed, the AC algorithm is asked to give back the delta domains.

My solver in OCaml is a \textit{functor} taking in parameter a module of type \textit{Arc\_consistency}.

The solver has two public functions:
\begin{minted}{ocaml}

  module type Solver = sig
    module DLL = DoublyLinkedList

    val initialization : ?verbose:bool -> string Constraint.graph -> unit

    val find_solution :
      ?debug:bool ->
      ?count_only:bool ->
      ?only_valid:bool ->
      ?verbose:bool ->
      ?one_sol:bool ->
      unit ->
      unit
  end

\end{minted}

These function aim to initiate the problem inside the solver taking a graph $g$ in entry, and to find one or all the solutions obtainable from $g$. All the optional argument of the \textit{find\_solution} method are detailed in \cref{sec:arg}.

Inside the functor we can find all the auxiliary attributes and functions allowing to solve the given problem.

\begin{minted}{ocaml}
  type 'a stack_type :
    (string AC.stack_operation * string DLL.node) option Stack.t
  val backtrack_mem : 'a stack_type 
  val stack_op : 'a stack_type
  val remove_by_node : ?verbose:bool -> string DLL.node -> unit
  val propagation_remove_by_node : ?verbose:bool -> string DLL.node -> unit
  val propagation_select_by_node : ?verbose:bool -> string DLL.node -> unit
  val back_track : unit -> unit
\end{minted}

\paragraph{stack\_op:} is the stack containing all the operation made inside the Arc-Consistency algorithm.

\paragraph{backtrack\_mem:} is the stack containing all the pointers to a previous state in the exploration tree in order to backtrack.

\paragraph{remove\_by\_node:} when we remove a value $v_i$ from a domain $D_i$, we call the Arc-Consistency algorithm passed to the solver \textit{functor} and we add to the stack of undo operation the delta domain the set of values to remove for propagation in a second moment. Note that if after remove $v_i$ from $D_i$, we throw the \textit{Empty\_domain} exception which will be caught in order to backtrack and find other solutions.

\paragraph{propagation\_remove\_by\_node:} this is an recursive function which propagates the deletion of a value $v_i$ which keep to remove all the value inside the delta domain until it is not empty.

\paragraph{propagation\_select\_by\_node:} is selected is a function that calls the \textit{propagation\_remove\_by\_node} for all the values inside $D_i$ that are different from the current value $v_i$. At each selection of a value $v_i$, we add to the \textit{backtrack\_mem} a pointer to the actual state of the solver in order to get it back during the backtrack step.

\subsection{The Arc Consistency Algorithms}

% \subsubsection{\ac{3}}

% \subsubsection{\ac{4}}

% \subsubsection{\ac{6}}

% \subsubsection{\ac{2001}}

% \section{The parser}

% \section{Benchmark}

% \subsection{AllIntervalSeries problem}
% \subsubsection{Stats}

% \plotProblem{AllIntervalSeries}

% \subsubsection{Generation of the problem}
% \label{sec:allIntGen}


% \subsection{Queens problem}
% \subsubsection{Stats}


% \plotProblem{Queens}

% \subsubsection{Generation of the problem}

% \section{Run the project}
% \label{sec:arg}

% \section{Conclusion}
% \nocite{*}
% \section{References}
% \printbibliography[heading=none]
% \newpage

\end{document}
